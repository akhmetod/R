\documentclass[12,]{article}
\usepackage{lmodern}
\usepackage{amssymb,amsmath}
\usepackage{ifxetex,ifluatex}
\usepackage{fixltx2e} % provides \textsubscript
\ifnum 0\ifxetex 1\fi\ifluatex 1\fi=0 % if pdftex
  \usepackage[T1]{fontenc}
  \usepackage[utf8]{inputenc}
\else % if luatex or xelatex
  \ifxetex
    \usepackage{mathspec}
  \else
    \usepackage{fontspec}
  \fi
  \defaultfontfeatures{Ligatures=TeX,Scale=MatchLowercase}
\fi
% use upquote if available, for straight quotes in verbatim environments
\IfFileExists{upquote.sty}{\usepackage{upquote}}{}
% use microtype if available
\IfFileExists{microtype.sty}{%
\usepackage{microtype}
\UseMicrotypeSet[protrusion]{basicmath} % disable protrusion for tt fonts
}{}
\usepackage[margin=1in]{geometry}
\usepackage{hyperref}
\hypersetup{unicode=true,
            pdftitle={Programmieren mit R: Seminararbeit 2},
            pdfauthor={Daniyar Akhmetov (5127348); Marcelo Rainho Avila (4679876); Xuan Son Le (4669361)},
            pdfborder={0 0 0},
            breaklinks=true}
\urlstyle{same}  % don't use monospace font for urls
\usepackage{color}
\usepackage{fancyvrb}
\newcommand{\VerbBar}{|}
\newcommand{\VERB}{\Verb[commandchars=\\\{\}]}
\DefineVerbatimEnvironment{Highlighting}{Verbatim}{commandchars=\\\{\}}
% Add ',fontsize=\small' for more characters per line
\usepackage{framed}
\definecolor{shadecolor}{RGB}{248,248,248}
\newenvironment{Shaded}{\begin{snugshade}}{\end{snugshade}}
\newcommand{\KeywordTok}[1]{\textcolor[rgb]{0.13,0.29,0.53}{\textbf{#1}}}
\newcommand{\DataTypeTok}[1]{\textcolor[rgb]{0.13,0.29,0.53}{#1}}
\newcommand{\DecValTok}[1]{\textcolor[rgb]{0.00,0.00,0.81}{#1}}
\newcommand{\BaseNTok}[1]{\textcolor[rgb]{0.00,0.00,0.81}{#1}}
\newcommand{\FloatTok}[1]{\textcolor[rgb]{0.00,0.00,0.81}{#1}}
\newcommand{\ConstantTok}[1]{\textcolor[rgb]{0.00,0.00,0.00}{#1}}
\newcommand{\CharTok}[1]{\textcolor[rgb]{0.31,0.60,0.02}{#1}}
\newcommand{\SpecialCharTok}[1]{\textcolor[rgb]{0.00,0.00,0.00}{#1}}
\newcommand{\StringTok}[1]{\textcolor[rgb]{0.31,0.60,0.02}{#1}}
\newcommand{\VerbatimStringTok}[1]{\textcolor[rgb]{0.31,0.60,0.02}{#1}}
\newcommand{\SpecialStringTok}[1]{\textcolor[rgb]{0.31,0.60,0.02}{#1}}
\newcommand{\ImportTok}[1]{#1}
\newcommand{\CommentTok}[1]{\textcolor[rgb]{0.56,0.35,0.01}{\textit{#1}}}
\newcommand{\DocumentationTok}[1]{\textcolor[rgb]{0.56,0.35,0.01}{\textbf{\textit{#1}}}}
\newcommand{\AnnotationTok}[1]{\textcolor[rgb]{0.56,0.35,0.01}{\textbf{\textit{#1}}}}
\newcommand{\CommentVarTok}[1]{\textcolor[rgb]{0.56,0.35,0.01}{\textbf{\textit{#1}}}}
\newcommand{\OtherTok}[1]{\textcolor[rgb]{0.56,0.35,0.01}{#1}}
\newcommand{\FunctionTok}[1]{\textcolor[rgb]{0.00,0.00,0.00}{#1}}
\newcommand{\VariableTok}[1]{\textcolor[rgb]{0.00,0.00,0.00}{#1}}
\newcommand{\ControlFlowTok}[1]{\textcolor[rgb]{0.13,0.29,0.53}{\textbf{#1}}}
\newcommand{\OperatorTok}[1]{\textcolor[rgb]{0.81,0.36,0.00}{\textbf{#1}}}
\newcommand{\BuiltInTok}[1]{#1}
\newcommand{\ExtensionTok}[1]{#1}
\newcommand{\PreprocessorTok}[1]{\textcolor[rgb]{0.56,0.35,0.01}{\textit{#1}}}
\newcommand{\AttributeTok}[1]{\textcolor[rgb]{0.77,0.63,0.00}{#1}}
\newcommand{\RegionMarkerTok}[1]{#1}
\newcommand{\InformationTok}[1]{\textcolor[rgb]{0.56,0.35,0.01}{\textbf{\textit{#1}}}}
\newcommand{\WarningTok}[1]{\textcolor[rgb]{0.56,0.35,0.01}{\textbf{\textit{#1}}}}
\newcommand{\AlertTok}[1]{\textcolor[rgb]{0.94,0.16,0.16}{#1}}
\newcommand{\ErrorTok}[1]{\textcolor[rgb]{0.64,0.00,0.00}{\textbf{#1}}}
\newcommand{\NormalTok}[1]{#1}
\usepackage{graphicx,grffile}
\makeatletter
\def\maxwidth{\ifdim\Gin@nat@width>\linewidth\linewidth\else\Gin@nat@width\fi}
\def\maxheight{\ifdim\Gin@nat@height>\textheight\textheight\else\Gin@nat@height\fi}
\makeatother
% Scale images if necessary, so that they will not overflow the page
% margins by default, and it is still possible to overwrite the defaults
% using explicit options in \includegraphics[width, height, ...]{}
\setkeys{Gin}{width=\maxwidth,height=\maxheight,keepaspectratio}
\IfFileExists{parskip.sty}{%
\usepackage{parskip}
}{% else
\setlength{\parindent}{0pt}
\setlength{\parskip}{6pt plus 2pt minus 1pt}
}
\setlength{\emergencystretch}{3em}  % prevent overfull lines
\providecommand{\tightlist}{%
  \setlength{\itemsep}{0pt}\setlength{\parskip}{0pt}}
\setcounter{secnumdepth}{5}
% Redefines (sub)paragraphs to behave more like sections
\ifx\paragraph\undefined\else
\let\oldparagraph\paragraph
\renewcommand{\paragraph}[1]{\oldparagraph{#1}\mbox{}}
\fi
\ifx\subparagraph\undefined\else
\let\oldsubparagraph\subparagraph
\renewcommand{\subparagraph}[1]{\oldsubparagraph{#1}\mbox{}}
\fi

%%% Use protect on footnotes to avoid problems with footnotes in titles
\let\rmarkdownfootnote\footnote%
\def\footnote{\protect\rmarkdownfootnote}

%%% Change title format to be more compact
\usepackage{titling}

% Create subtitle command for use in maketitle
\newcommand{\subtitle}[1]{
  \posttitle{
    \begin{center}\large#1\end{center}
    }
}

\setlength{\droptitle}{-2em}
  \title{Programmieren mit R: Seminararbeit 2}
  \pretitle{\vspace{\droptitle}\centering\huge}
  \posttitle{\par}
  \author{Daniyar Akhmetov (5127348) \\ Marcelo Rainho Avila (4679876) \\ Xuan Son Le (4669361)}
  \preauthor{\centering\large\emph}
  \postauthor{\par}
  \predate{\centering\large\emph}
  \postdate{\par}
  \date{Abgabedatum: 19.12.2017}


\begin{document}
\maketitle

{
\setcounter{tocdepth}{3}
\tableofcontents
}
\newpage

\section{\texorpdfstring{Part I: \emph{Functions} (15
points)}{Part I: Functions (15 points)}}\label{part-i-functions-15-points}

\subsection{Functions I:}\label{functions-i}

Define a function which given an atomic vector x as argument, returns x
after removing missing values

\begin{Shaded}
\begin{Highlighting}[]
\NormalTok{dropNa <-}\StringTok{ }\ControlFlowTok{function}\NormalTok{(x) \{}
  \CommentTok{# takes an atomic vector as an argument and returns it without missing values}
\NormalTok{  x[}\OperatorTok{!}\KeywordTok{is.na}\NormalTok{(x)]}
\NormalTok{\}}
\end{Highlighting}
\end{Shaded}

\begin{Shaded}
\begin{Highlighting}[]
\KeywordTok{all.equal}\NormalTok{(}\KeywordTok{dropNa}\NormalTok{(}\KeywordTok{c}\NormalTok{(}\DecValTok{1}\NormalTok{, }\DecValTok{2}\NormalTok{, }\DecValTok{3}\NormalTok{, }\OtherTok{NA}\NormalTok{, }\DecValTok{1}\NormalTok{, }\DecValTok{2}\NormalTok{, }\DecValTok{3}\NormalTok{)), }\KeywordTok{c}\NormalTok{(}\DecValTok{1}\NormalTok{, }\DecValTok{2}\NormalTok{, }\DecValTok{3}\NormalTok{, }\DecValTok{1}\NormalTok{, }\DecValTok{2}\NormalTok{, }\DecValTok{3}\NormalTok{))}
\end{Highlighting}
\end{Shaded}

\begin{verbatim}
## [1] TRUE
\end{verbatim}

\subsection{Functions II:}\label{functions-ii}

\emph{Part I:} Write a function meanVarSdSe that takes a numeric vector
x as argument. The function should return a named numeric vector that
contains the mean, the variance, the standard deviation and the standard
error of x.

\begin{Shaded}
\begin{Highlighting}[]
\NormalTok{meanVarSdSe <-}\StringTok{ }\ControlFlowTok{function}\NormalTok{(x)\{}
  \CommentTok{# takes an numeric vector as argument and returns a named numeric vector }
  \CommentTok{# containing its mean, variance, standard deviation and standard error }
  \KeywordTok{c}\NormalTok{(}\DataTypeTok{mean =} \KeywordTok{mean}\NormalTok{(x),}
    \DataTypeTok{var =} \KeywordTok{var}\NormalTok{(x),}
    \DataTypeTok{sd =} \KeywordTok{sd}\NormalTok{(x),}
    \DataTypeTok{se =} \KeywordTok{sd}\NormalTok{(x) }\OperatorTok{/}\StringTok{ }\KeywordTok{sqrt}\NormalTok{(}\KeywordTok{length}\NormalTok{(x))}
\NormalTok{  )}
\NormalTok{\}}

\CommentTok{# test}
\NormalTok{x <-}\StringTok{ }\DecValTok{1}\OperatorTok{:}\DecValTok{100}
\KeywordTok{meanVarSdSe}\NormalTok{(x)}
\end{Highlighting}
\end{Shaded}

\begin{verbatim}
##       mean        var         sd         se 
##  50.500000 841.666667  29.011492   2.901149
\end{verbatim}

\vspace{1cm}

\emph{Part II:} Look at the following code sequence. What result do you
expect?

\begin{Shaded}
\begin{Highlighting}[]
\NormalTok{x <-}\StringTok{ }\KeywordTok{c}\NormalTok{(}\OtherTok{NA}\NormalTok{, }\DecValTok{1}\OperatorTok{:}\DecValTok{100}\NormalTok{)}
\KeywordTok{meanVarSdSe}\NormalTok{(x)}
\end{Highlighting}
\end{Shaded}

The code returns NA values for each statistic computed, which is the
output of each function when using the default (FALSE) argument for
\texttt{na.rm}.

\begin{Shaded}
\begin{Highlighting}[]
\NormalTok{meanVarSdSe <-}\StringTok{ }\ControlFlowTok{function}\NormalTok{(x, ...)\{}
  \CommentTok{# computs mean, variance, standard deviation and standard error}
  \CommentTok{#}
  \CommentTok{# Args:}
  \CommentTok{#   x: a numeric vector}
  \CommentTok{#}
  \CommentTok{# Returns:}
  \CommentTok{#   mean, variance, standard deviation and standard error of input vector}
  \KeywordTok{c}\NormalTok{(}\DataTypeTok{mean =} \KeywordTok{mean}\NormalTok{(x, ...),}
    \DataTypeTok{var =} \KeywordTok{var}\NormalTok{(x, ...),}
    \DataTypeTok{sd =} \KeywordTok{sd}\NormalTok{(x, ...),}
    \DataTypeTok{se =} \KeywordTok{sd}\NormalTok{(x, ...) }\OperatorTok{/}\StringTok{ }\KeywordTok{sqrt}\NormalTok{(}\KeywordTok{length}\NormalTok{(}\KeywordTok{which}\NormalTok{(}\OperatorTok{!}\KeywordTok{is.na}\NormalTok{(x))))}
\NormalTok{  )}
\NormalTok{\}}

\CommentTok{# test}
\KeywordTok{meanVarSdSe}\NormalTok{(x, }\DataTypeTok{na.rm =} \OtherTok{TRUE}\NormalTok{)}
\end{Highlighting}
\end{Shaded}

\begin{verbatim}
##       mean        var         sd         se 
##  50.500000 841.666667  29.011492   2.901149
\end{verbatim}

\vspace{1cm}

\emph{Part III:} Write an alternative version of meanVarSdSe in which
you make use of the function definition \textbf{dropNa} from the above
exercise.

\begin{Shaded}
\begin{Highlighting}[]
\NormalTok{meanVarSdSe <-}\StringTok{ }\ControlFlowTok{function}\NormalTok{(x)\{}
  \CommentTok{# computs mean, variance, standard deviation and standard error while using }
  \CommentTok{# dropNa function}
  \CommentTok{#}
  \CommentTok{# Args:}
  \CommentTok{#   x: a numeric vector}
  \CommentTok{#}
  \CommentTok{# Returns:}
  \CommentTok{#   mean, variance, standard deviation and standard error of input vector}
\NormalTok{  dropedNa <-}\StringTok{ }\KeywordTok{dropNa}\NormalTok{(x)}
  \KeywordTok{c}\NormalTok{(}\DataTypeTok{mean =} \KeywordTok{mean}\NormalTok{(dropedNa),}
    \DataTypeTok{var =} \KeywordTok{var}\NormalTok{(dropedNa),}
    \DataTypeTok{sd =} \KeywordTok{sd}\NormalTok{(dropedNa),}
    \DataTypeTok{se =} \KeywordTok{sd}\NormalTok{(dropedNa) }\OperatorTok{/}\StringTok{ }\KeywordTok{sqrt}\NormalTok{(}\KeywordTok{length}\NormalTok{(dropedNa))}
\NormalTok{  )}
\NormalTok{\}}

\CommentTok{# test}
\KeywordTok{meanVarSdSe}\NormalTok{(}\KeywordTok{c}\NormalTok{(x, }\OtherTok{NA}\NormalTok{))}
\end{Highlighting}
\end{Shaded}

\begin{verbatim}
##       mean        var         sd         se 
##  50.500000 841.666667  29.011492   2.901149
\end{verbatim}

\subsection{Functions III:}\label{functions-iii}

Write an infix function \%or\% that behaves like the logical operator
\textbar{}

\begin{Shaded}
\begin{Highlighting}[]
\StringTok{"%or%"}\NormalTok{ <-}\StringTok{ }\ControlFlowTok{function}\NormalTok{(x,y) \{}
  \CommentTok{# logical operator OR:}
  \CommentTok{# TRUE OR TRUE = TRUE}
  \CommentTok{# TRUE OR FALSE = TRUE}
  \CommentTok{# FALSE OR TRUE = TRUE}
  \CommentTok{# FALSE OR FALSE = FALSE}
  
  \KeywordTok{ifelse}\NormalTok{(x }\OperatorTok{==}\StringTok{ }\OtherTok{TRUE}\NormalTok{, }\OtherTok{TRUE}\NormalTok{,}
         \KeywordTok{ifelse}\NormalTok{(y }\OperatorTok{==}\StringTok{ }\OtherTok{TRUE}\NormalTok{, }\OtherTok{TRUE}\NormalTok{, }\OtherTok{FALSE}\NormalTok{))}
\NormalTok{\}}

\KeywordTok{c}\NormalTok{(}\OtherTok{TRUE}\NormalTok{, }\OtherTok{FALSE}\NormalTok{, }\OtherTok{TRUE}\NormalTok{, }\OtherTok{FALSE}\NormalTok{) }\OperatorTok\StringTok{ }\KeywordTok{c}\NormalTok{(}\OtherTok{TRUE}\NormalTok{, }\OtherTok{TRUE}\NormalTok{, }\OtherTok{FALSE}\NormalTok{, }\OtherTok{FALSE}\NormalTok{)}
\end{Highlighting}
\end{Shaded}

\begin{verbatim}
## [1]  TRUE  TRUE  TRUE FALSE
\end{verbatim}

\section{\texorpdfstring{Part II: \emph{Scoping and related topics } (15
points)}{Part II: Scoping and related topics  (15 points)}}\label{part-ii-scoping-and-related-topics-15-points}

\emph{Scoping I:} Explain the results of the three function calls

\begin{Shaded}
\begin{Highlighting}[]
\NormalTok{x <-}\StringTok{ }\DecValTok{5}
\NormalTok{y <-}\StringTok{ }\DecValTok{7}
\NormalTok{f <-}\StringTok{ }\ControlFlowTok{function}\NormalTok{() x }\OperatorTok{*}\StringTok{ }\NormalTok{y}
\NormalTok{g <-}\StringTok{ }\ControlFlowTok{function}\NormalTok{(}\DataTypeTok{x =} \DecValTok{2}\NormalTok{, }\DataTypeTok{y =}\NormalTok{ x) x }\OperatorTok{*}\StringTok{ }\NormalTok{y}
\KeywordTok{f}\NormalTok{() }\CommentTok{# call 1}
\end{Highlighting}
\end{Shaded}

\begin{verbatim}
## [1] 35
\end{verbatim}

\begin{Shaded}
\begin{Highlighting}[]
\KeywordTok{g}\NormalTok{() }\CommentTok{# call 2}
\end{Highlighting}
\end{Shaded}

\begin{verbatim}
## [1] 4
\end{verbatim}

\begin{Shaded}
\begin{Highlighting}[]
\KeywordTok{g}\NormalTok{(}\DataTypeTok{y=}\NormalTok{x) }\CommentTok{#call3}
\end{Highlighting}
\end{Shaded}

\begin{verbatim}
## [1] 10
\end{verbatim}

x and y are originally assigned to the value 5 and 7 respectively.\\
Function f does not require any arguments and is defined as the product
of x and y, which is 5*7 = 35 (see call 1).\\
Function g has two arguments, which are x with the new value of 2 and y
is assigned equally to x. Function g is also defined as the product of x
and y. In call 2, function g is calles without any arguments. In this
function, x is equal to 2 and y is equal to x. The nearest x to y has
the value of 2. So the product is 4. In call 3, x is not called, so x
will get the original value, which is 5. y is assigned equally to x. The
nearest x to y has the value of 2. So the product is 10 in this case.

\emph{Scoping II:} Why and how does the following code work?

\begin{Shaded}
\begin{Highlighting}[]
\NormalTok{t <-}\StringTok{ }\KeywordTok{matrix}\NormalTok{(}\DecValTok{1}\OperatorTok{:}\DecValTok{6}\NormalTok{, }\DataTypeTok{ncol =} \DecValTok{3}\NormalTok{, }\DataTypeTok{byrow =} \OtherTok{TRUE}\NormalTok{) }
\NormalTok{t}
\end{Highlighting}
\end{Shaded}

\begin{verbatim}
##      [,1] [,2] [,3]
## [1,]    1    2    3
## [2,]    4    5    6
\end{verbatim}

\begin{Shaded}
\begin{Highlighting}[]
\KeywordTok{t}\NormalTok{(t)}
\end{Highlighting}
\end{Shaded}

\begin{verbatim}
##      [,1] [,2]
## [1,]    1    4
## [2,]    2    5
## [3,]    3    6
\end{verbatim}

t is a matrix with 3 columns and 2 rows. The values from 1 to 6 are
filled by rows. t is also a default function in R, which returns the
transpose of a matrix or data frame. So calling t(t) is actually
transposing the defined matrix t.

\emph{Scoping III:} Why do the results of t(T) and t(t) differ?

\begin{Shaded}
\begin{Highlighting}[]
\NormalTok{t <-}\StringTok{ }\ControlFlowTok{function}\NormalTok{(...) }\KeywordTok{matrix}\NormalTok{(...)}
\NormalTok{T <-}\StringTok{ }\KeywordTok{t}\NormalTok{(}\DecValTok{1}\OperatorTok{:}\DecValTok{6}\NormalTok{, }\DataTypeTok{ncol =} \DecValTok{3}\NormalTok{, }\DataTypeTok{byrow =} \OtherTok{TRUE}\NormalTok{)}
\KeywordTok{t}\NormalTok{(T)}
\end{Highlighting}
\end{Shaded}

\begin{verbatim}
##      [,1]
## [1,]    1
## [2,]    4
## [3,]    2
## [4,]    5
## [5,]    3
## [6,]    6
\end{verbatim}

In the first case, t is defined as a specific function, which has the
same function as a matrix(..) function. T is a matrix with 3 columns and
2 rows. The values from 1 to 6 are also filled by rows. So calling t(T)
is the same as calling matrix(T).

\begin{Shaded}
\begin{Highlighting}[]
\NormalTok{t <-}\StringTok{ }\ControlFlowTok{function}\NormalTok{(...) }\KeywordTok{matrix}\NormalTok{(...)}
\NormalTok{t <-}\StringTok{ }\KeywordTok{t}\NormalTok{(}\DecValTok{1}\OperatorTok{:}\DecValTok{6}\NormalTok{, }\DataTypeTok{ncol =} \DecValTok{3}\NormalTok{, }\DataTypeTok{byrow =} \OtherTok{TRUE}\NormalTok{) }
\KeywordTok{t}\NormalTok{(t)}
\end{Highlighting}
\end{Shaded}

\begin{verbatim}
##      [,1] [,2]
## [1,]    1    4
## [2,]    2    5
## [3,]    3    6
\end{verbatim}

In the second case, t is also defined as a matrix() function. But then
being redefined as a matrix. So R will understand t(t) as a transpose of
a matrix.

\emph{Dynamic lookup:} Explain the results of the five function calls
and why the rm function in line 1 is important.

\begin{Shaded}
\begin{Highlighting}[]
\KeywordTok{rm}\NormalTok{(}\DataTypeTok{list =} \KeywordTok{ls}\NormalTok{(}\DataTypeTok{all.names =} \OtherTok{TRUE}\NormalTok{))}
\NormalTok{f <-}\StringTok{ }\ControlFlowTok{function}\NormalTok{(x, }\DataTypeTok{y =}\NormalTok{ x }\OperatorTok{+}\StringTok{ }\DecValTok{1}\NormalTok{) x }\OperatorTok{+}\StringTok{ }\NormalTok{y }
\NormalTok{x <-}\StringTok{ }\DecValTok{3}
\KeywordTok{f}\NormalTok{(}\DecValTok{2}\NormalTok{) }\CommentTok{# call 1}
\end{Highlighting}
\end{Shaded}

\begin{verbatim}
## [1] 5
\end{verbatim}

\begin{Shaded}
\begin{Highlighting}[]
\NormalTok{x <-}\StringTok{ }\DecValTok{5}
\KeywordTok{f}\NormalTok{(}\DecValTok{2}\NormalTok{) }\CommentTok{# call 2}
\end{Highlighting}
\end{Shaded}

\begin{verbatim}
## [1] 5
\end{verbatim}

\begin{Shaded}
\begin{Highlighting}[]
\NormalTok{f <-}\StringTok{ }\ControlFlowTok{function}\NormalTok{(}\DataTypeTok{y =}\NormalTok{ x }\OperatorTok{+}\StringTok{ }\DecValTok{1}\NormalTok{) x }\OperatorTok{+}\StringTok{ }\NormalTok{y }
\NormalTok{x <-}\StringTok{ }\DecValTok{3}
\KeywordTok{f}\NormalTok{(}\DecValTok{2}\NormalTok{) }\CommentTok{# call 3}
\end{Highlighting}
\end{Shaded}

\begin{verbatim}
## [1] 5
\end{verbatim}

\begin{Shaded}
\begin{Highlighting}[]
\NormalTok{x <-}\StringTok{ }\DecValTok{5}
\KeywordTok{f}\NormalTok{(}\DecValTok{2}\NormalTok{) }\CommentTok{# call 4 }
\end{Highlighting}
\end{Shaded}

\begin{verbatim}
## [1] 7
\end{verbatim}

\begin{Shaded}
\begin{Highlighting}[]
\KeywordTok{f}\NormalTok{() }\CommentTok{# call 5}
\end{Highlighting}
\end{Shaded}

\begin{verbatim}
## [1] 11
\end{verbatim}

No idea xD


\end{document}
