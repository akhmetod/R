\documentclass[12,]{article}
\usepackage{lmodern}
\usepackage{amssymb,amsmath}
\usepackage{ifxetex,ifluatex}
\usepackage{fixltx2e} % provides \textsubscript
\ifnum 0\ifxetex 1\fi\ifluatex 1\fi=0 % if pdftex
  \usepackage[T1]{fontenc}
  \usepackage[utf8]{inputenc}
\else % if luatex or xelatex
  \ifxetex
    \usepackage{mathspec}
  \else
    \usepackage{fontspec}
  \fi
  \defaultfontfeatures{Ligatures=TeX,Scale=MatchLowercase}
\fi
% use upquote if available, for straight quotes in verbatim environments
\IfFileExists{upquote.sty}{\usepackage{upquote}}{}
% use microtype if available
\IfFileExists{microtype.sty}{%
\usepackage{microtype}
\UseMicrotypeSet[protrusion]{basicmath} % disable protrusion for tt fonts
}{}
\usepackage[margin=1in]{geometry}
\usepackage{hyperref}
\hypersetup{unicode=true,
            pdftitle={Programmieren mit R: Seminararbeit 2},
            pdfauthor={Daniyar Akhmetov (5127348); Marcelo Rainho Avila (4679876); Xuan Son Le (4669361)},
            pdfborder={0 0 0},
            breaklinks=true}
\urlstyle{same}  % don't use monospace font for urls
\usepackage{color}
\usepackage{fancyvrb}
\newcommand{\VerbBar}{|}
\newcommand{\VERB}{\Verb[commandchars=\\\{\}]}
\DefineVerbatimEnvironment{Highlighting}{Verbatim}{commandchars=\\\{\}}
% Add ',fontsize=\small' for more characters per line
\usepackage{framed}
\definecolor{shadecolor}{RGB}{248,248,248}
\newenvironment{Shaded}{\begin{snugshade}}{\end{snugshade}}
\newcommand{\KeywordTok}[1]{\textcolor[rgb]{0.13,0.29,0.53}{\textbf{#1}}}
\newcommand{\DataTypeTok}[1]{\textcolor[rgb]{0.13,0.29,0.53}{#1}}
\newcommand{\DecValTok}[1]{\textcolor[rgb]{0.00,0.00,0.81}{#1}}
\newcommand{\BaseNTok}[1]{\textcolor[rgb]{0.00,0.00,0.81}{#1}}
\newcommand{\FloatTok}[1]{\textcolor[rgb]{0.00,0.00,0.81}{#1}}
\newcommand{\ConstantTok}[1]{\textcolor[rgb]{0.00,0.00,0.00}{#1}}
\newcommand{\CharTok}[1]{\textcolor[rgb]{0.31,0.60,0.02}{#1}}
\newcommand{\SpecialCharTok}[1]{\textcolor[rgb]{0.00,0.00,0.00}{#1}}
\newcommand{\StringTok}[1]{\textcolor[rgb]{0.31,0.60,0.02}{#1}}
\newcommand{\VerbatimStringTok}[1]{\textcolor[rgb]{0.31,0.60,0.02}{#1}}
\newcommand{\SpecialStringTok}[1]{\textcolor[rgb]{0.31,0.60,0.02}{#1}}
\newcommand{\ImportTok}[1]{#1}
\newcommand{\CommentTok}[1]{\textcolor[rgb]{0.56,0.35,0.01}{\textit{#1}}}
\newcommand{\DocumentationTok}[1]{\textcolor[rgb]{0.56,0.35,0.01}{\textbf{\textit{#1}}}}
\newcommand{\AnnotationTok}[1]{\textcolor[rgb]{0.56,0.35,0.01}{\textbf{\textit{#1}}}}
\newcommand{\CommentVarTok}[1]{\textcolor[rgb]{0.56,0.35,0.01}{\textbf{\textit{#1}}}}
\newcommand{\OtherTok}[1]{\textcolor[rgb]{0.56,0.35,0.01}{#1}}
\newcommand{\FunctionTok}[1]{\textcolor[rgb]{0.00,0.00,0.00}{#1}}
\newcommand{\VariableTok}[1]{\textcolor[rgb]{0.00,0.00,0.00}{#1}}
\newcommand{\ControlFlowTok}[1]{\textcolor[rgb]{0.13,0.29,0.53}{\textbf{#1}}}
\newcommand{\OperatorTok}[1]{\textcolor[rgb]{0.81,0.36,0.00}{\textbf{#1}}}
\newcommand{\BuiltInTok}[1]{#1}
\newcommand{\ExtensionTok}[1]{#1}
\newcommand{\PreprocessorTok}[1]{\textcolor[rgb]{0.56,0.35,0.01}{\textit{#1}}}
\newcommand{\AttributeTok}[1]{\textcolor[rgb]{0.77,0.63,0.00}{#1}}
\newcommand{\RegionMarkerTok}[1]{#1}
\newcommand{\InformationTok}[1]{\textcolor[rgb]{0.56,0.35,0.01}{\textbf{\textit{#1}}}}
\newcommand{\WarningTok}[1]{\textcolor[rgb]{0.56,0.35,0.01}{\textbf{\textit{#1}}}}
\newcommand{\AlertTok}[1]{\textcolor[rgb]{0.94,0.16,0.16}{#1}}
\newcommand{\ErrorTok}[1]{\textcolor[rgb]{0.64,0.00,0.00}{\textbf{#1}}}
\newcommand{\NormalTok}[1]{#1}
\usepackage{graphicx,grffile}
\makeatletter
\def\maxwidth{\ifdim\Gin@nat@width>\linewidth\linewidth\else\Gin@nat@width\fi}
\def\maxheight{\ifdim\Gin@nat@height>\textheight\textheight\else\Gin@nat@height\fi}
\makeatother
% Scale images if necessary, so that they will not overflow the page
% margins by default, and it is still possible to overwrite the defaults
% using explicit options in \includegraphics[width, height, ...]{}
\setkeys{Gin}{width=\maxwidth,height=\maxheight,keepaspectratio}
\IfFileExists{parskip.sty}{%
\usepackage{parskip}
}{% else
\setlength{\parindent}{0pt}
\setlength{\parskip}{6pt plus 2pt minus 1pt}
}
\setlength{\emergencystretch}{3em}  % prevent overfull lines
\providecommand{\tightlist}{%
  \setlength{\itemsep}{0pt}\setlength{\parskip}{0pt}}
\setcounter{secnumdepth}{5}
% Redefines (sub)paragraphs to behave more like sections
\ifx\paragraph\undefined\else
\let\oldparagraph\paragraph
\renewcommand{\paragraph}[1]{\oldparagraph{#1}\mbox{}}
\fi
\ifx\subparagraph\undefined\else
\let\oldsubparagraph\subparagraph
\renewcommand{\subparagraph}[1]{\oldsubparagraph{#1}\mbox{}}
\fi

%%% Use protect on footnotes to avoid problems with footnotes in titles
\let\rmarkdownfootnote\footnote%
\def\footnote{\protect\rmarkdownfootnote}

%%% Change title format to be more compact
\usepackage{titling}

% Create subtitle command for use in maketitle
\newcommand{\subtitle}[1]{
  \posttitle{
    \begin{center}\large#1\end{center}
    }
}

\setlength{\droptitle}{-2em}
  \title{Programmieren mit R: Seminararbeit 2}
  \pretitle{\vspace{\droptitle}\centering\huge}
  \posttitle{\par}
  \author{Daniyar Akhmetov (5127348) \\ Marcelo Rainho Avila (4679876) \\ Xuan Son Le (4669361)}
  \preauthor{\centering\large\emph}
  \postauthor{\par}
  \predate{\centering\large\emph}
  \postdate{\par}
  \date{Abgabedatum: 19.12.2017}


\begin{document}
\maketitle

{
\setcounter{tocdepth}{3}
\tableofcontents
}
\newpage

\section{\texorpdfstring{Part I: \emph{Functions} (15
points)}{Part I: Functions (15 points)}}\label{part-i-functions-15-points}

\subsection{Functions I:}\label{functions-i}

Define a function which given an atomic vector x as argument, returns x
after removing missing values

\begin{Shaded}
\begin{Highlighting}[]
\NormalTok{dropNa <-}\StringTok{ }\ControlFlowTok{function}\NormalTok{(x) \{}
  \CommentTok{# takes an atomic vector as an argument and returns it without missing values}
  \CommentTok{#}
  \CommentTok{# Args:}
  \CommentTok{#   x: a numeric vector}
  \CommentTok{#}
  \CommentTok{# Returns:}
  \CommentTok{#   atomic vector without NA values}
\NormalTok{  x[}\OperatorTok{!}\KeywordTok{is.na}\NormalTok{(x)]}
\NormalTok{\}}
\end{Highlighting}
\end{Shaded}

\begin{Shaded}
\begin{Highlighting}[]
\KeywordTok{all.equal}\NormalTok{(}\KeywordTok{dropNa}\NormalTok{(}\KeywordTok{c}\NormalTok{(}\DecValTok{1}\NormalTok{, }\DecValTok{2}\NormalTok{, }\DecValTok{3}\NormalTok{, }\OtherTok{NA}\NormalTok{, }\DecValTok{1}\NormalTok{, }\DecValTok{2}\NormalTok{, }\DecValTok{3}\NormalTok{)), }\KeywordTok{c}\NormalTok{(}\DecValTok{1}\NormalTok{, }\DecValTok{2}\NormalTok{, }\DecValTok{3}\NormalTok{, }\DecValTok{1}\NormalTok{, }\DecValTok{2}\NormalTok{, }\DecValTok{3}\NormalTok{))}
\end{Highlighting}
\end{Shaded}

\begin{verbatim}
## [1] TRUE
\end{verbatim}

\subsection{Functions II:}\label{functions-ii}

\subsubsection*{Part I}\label{part-i}
\addcontentsline{toc}{subsubsection}{Part I}

Write a function meanVarSdSe that takes a numeric vector x as argument.
The function should return a named numeric vector that contains the
mean, the variance, the standard deviation and the standard error of x.

\begin{Shaded}
\begin{Highlighting}[]
\NormalTok{meanVarSdSe <-}\StringTok{ }\ControlFlowTok{function}\NormalTok{(x)\{}
  \CommentTok{# computes mean, variance, standard deviation and standard error }
  \CommentTok{#}
  \CommentTok{# Args:}
  \CommentTok{#   x: a numeric vector}
  \CommentTok{#}
  \CommentTok{# Returns:}
  \CommentTok{#   mean, variance, standard deviation and standard error of input vector}
  \KeywordTok{c}\NormalTok{(}\DataTypeTok{mean =} \KeywordTok{mean}\NormalTok{(x),}
    \DataTypeTok{var =} \KeywordTok{var}\NormalTok{(x),}
    \DataTypeTok{sd =} \KeywordTok{sd}\NormalTok{(x),}
    \DataTypeTok{se =} \KeywordTok{sd}\NormalTok{(x) }\OperatorTok{/}\StringTok{ }\KeywordTok{sqrt}\NormalTok{(}\KeywordTok{length}\NormalTok{(x))}
\NormalTok{  )}
\NormalTok{\}}

\CommentTok{# test}
\NormalTok{x <-}\StringTok{ }\DecValTok{1}\OperatorTok{:}\DecValTok{100}
\KeywordTok{meanVarSdSe}\NormalTok{(x)}
\end{Highlighting}
\end{Shaded}

\begin{verbatim}
##       mean        var         sd         se
##  50.500000 841.666667  29.011492   2.901149
\end{verbatim}

\subsubsection*{Part II}\label{part-ii}
\addcontentsline{toc}{subsubsection}{Part II}

Look at the following code sequence. What result do you expect?

\begin{Shaded}
\begin{Highlighting}[]
\NormalTok{x <-}\StringTok{ }\KeywordTok{c}\NormalTok{(}\OtherTok{NA}\NormalTok{, }\DecValTok{1}\OperatorTok{:}\DecValTok{100}\NormalTok{)}
\KeywordTok{meanVarSdSe}\NormalTok{(x)}
\end{Highlighting}
\end{Shaded}

The code returns NA values for each statistic computed, which is the
output of each function when using the default (FALSE) argument for
removing NA's (\texttt{na.rm}.)

\begin{Shaded}
\begin{Highlighting}[]
\NormalTok{meanVarSdSe <-}\StringTok{ }\ControlFlowTok{function}\NormalTok{(x, ...)\{}
  \CommentTok{# computes mean, variance, standard deviation and standard error}
  \CommentTok{#}
  \CommentTok{# Args:}
  \CommentTok{#   x: a numeric vector}
  \CommentTok{#}
  \CommentTok{# Returns:}
  \CommentTok{#   mean, variance, standard deviation and standard error of input vector}
  \KeywordTok{c}\NormalTok{(}\DataTypeTok{mean =} \KeywordTok{mean}\NormalTok{(x, ...),}
    \DataTypeTok{var =} \KeywordTok{var}\NormalTok{(x, ...),}
    \DataTypeTok{sd =} \KeywordTok{sd}\NormalTok{(x, ...),}
    \DataTypeTok{se =} \KeywordTok{sd}\NormalTok{(x, ...) }\OperatorTok{/}\StringTok{ }\KeywordTok{sqrt}\NormalTok{(}\KeywordTok{length}\NormalTok{(}\KeywordTok{which}\NormalTok{(}\OperatorTok{!}\KeywordTok{is.na}\NormalTok{(x))))}
\NormalTok{  )}
\NormalTok{\}}

\CommentTok{# test}
\KeywordTok{meanVarSdSe}\NormalTok{(x, }\DataTypeTok{na.rm =} \OtherTok{TRUE}\NormalTok{)}
\end{Highlighting}
\end{Shaded}

\begin{verbatim}
##       mean        var         sd         se
##  50.500000 841.666667  29.011492   2.901149
\end{verbatim}

\subsubsection*{Part III}\label{part-iii}
\addcontentsline{toc}{subsubsection}{Part III}

Write an alternative version of meanVarSdSe in which you make use of the
function definition \textbf{dropNa} from the above exercise.

\begin{Shaded}
\begin{Highlighting}[]
\NormalTok{meanVarSdSe <-}\StringTok{ }\ControlFlowTok{function}\NormalTok{(x, }\DataTypeTok{dropMissing =} \OtherTok{TRUE}\NormalTok{)\{}
  \CommentTok{# computs mean, variance, standard deviation and standard error while using }
  \CommentTok{# dropNa function per default}
  \CommentTok{#}
  \CommentTok{# Args:}
  \CommentTok{#   x: a numeric vector}
  \CommentTok{#}
  \CommentTok{# Returns:}
  \CommentTok{#   mean, variance, standard deviation and standard error of input vector}
  \ControlFlowTok{if}\NormalTok{ (dropMissing) \{}
\NormalTok{    x <-}\StringTok{ }\KeywordTok{dropNa}\NormalTok{(x)}
\NormalTok{  \}}
  \KeywordTok{c}\NormalTok{(}\DataTypeTok{mean =} \KeywordTok{mean}\NormalTok{(x),}
    \DataTypeTok{var =} \KeywordTok{var}\NormalTok{(x),}
    \DataTypeTok{sd =} \KeywordTok{sd}\NormalTok{(x),}
    \DataTypeTok{se =} \KeywordTok{sd}\NormalTok{(x) }\OperatorTok{/}\StringTok{ }\KeywordTok{sqrt}\NormalTok{(}\KeywordTok{length}\NormalTok{(x))}
\NormalTok{  )}
\NormalTok{\}}

\CommentTok{# test}
\KeywordTok{meanVarSdSe}\NormalTok{(}\KeywordTok{c}\NormalTok{(x, }\OtherTok{NA}\NormalTok{))}
\end{Highlighting}
\end{Shaded}

\begin{verbatim}
##       mean        var         sd         se
##  50.500000 841.666667  29.011492   2.901149
\end{verbatim}

\subsection{Functions III:}\label{functions-iii}

Write an infix function \%or\% that behaves like the logical operator
\textbar{}

\begin{Shaded}
\begin{Highlighting}[]
\StringTok{"%or%"}\NormalTok{ <-}\StringTok{ }\ControlFlowTok{function}\NormalTok{(x,y) \{}
  \CommentTok{# logical operator OR:}
  \CommentTok{# TRUE OR TRUE = TRUE}
  \CommentTok{# TRUE OR FALSE = TRUE}
  \CommentTok{# FALSE OR TRUE = TRUE}
  \CommentTok{# FALSE OR FALSE = FALSE}

  \KeywordTok{ifelse}\NormalTok{(x }\OperatorTok{==}\StringTok{ }\OtherTok{TRUE}\NormalTok{, }\OtherTok{TRUE}\NormalTok{,}
         \KeywordTok{ifelse}\NormalTok{(y }\OperatorTok{==}\StringTok{ }\OtherTok{TRUE}\NormalTok{, }\OtherTok{TRUE}\NormalTok{, }\OtherTok{FALSE}\NormalTok{))}
\NormalTok{\}}

\CommentTok{# test}
\KeywordTok{c}\NormalTok{(}\OtherTok{TRUE}\NormalTok{, }\OtherTok{FALSE}\NormalTok{, }\OtherTok{TRUE}\NormalTok{, }\OtherTok{FALSE}\NormalTok{) }\OperatorTok\StringTok{ }\KeywordTok{c}\NormalTok{(}\OtherTok{TRUE}\NormalTok{, }\OtherTok{TRUE}\NormalTok{, }\OtherTok{FALSE}\NormalTok{, }\OtherTok{FALSE}\NormalTok{)}
\end{Highlighting}
\end{Shaded}

\begin{verbatim}
## [1]  TRUE  TRUE  TRUE FALSE
\end{verbatim}

\section{\texorpdfstring{Part II: \emph{Scoping and related topics } (15
points)}{Part II: Scoping and related topics  (15 points)}}\label{part-ii-scoping-and-related-topics-15-points}

\subsection*{Scoping I}\label{scoping-i}
\addcontentsline{toc}{subsection}{Scoping I}

Explain the results of the three function calls

\begin{Shaded}
\begin{Highlighting}[]
\NormalTok{x <-}\StringTok{ }\DecValTok{5}
\NormalTok{y <-}\StringTok{ }\DecValTok{7}
\NormalTok{f <-}\StringTok{ }\ControlFlowTok{function}\NormalTok{() x }\OperatorTok{*}\StringTok{ }\NormalTok{y}
\NormalTok{g <-}\StringTok{ }\ControlFlowTok{function}\NormalTok{(}\DataTypeTok{x =} \DecValTok{2}\NormalTok{, }\DataTypeTok{y =}\NormalTok{ x) x }\OperatorTok{*}\StringTok{ }\NormalTok{y}
\end{Highlighting}
\end{Shaded}

\begin{Shaded}
\begin{Highlighting}[]
\KeywordTok{f}\NormalTok{()     }\CommentTok{# call 1}
\end{Highlighting}
\end{Shaded}

\begin{verbatim}
## [1] 35
\end{verbatim}

Function \texttt{f()} does not require any arguments and is defined as
the product of \texttt{x} and \texttt{y}. It raises an error if
\texttt{x} or \texttt{y} are not are not defined in the global (or any
other parent) environment.

\begin{Shaded}
\begin{Highlighting}[]
\KeywordTok{g}\NormalTok{()     }\CommentTok{# call 2}
\end{Highlighting}
\end{Shaded}

\begin{verbatim}
## [1] 4
\end{verbatim}

Function \texttt{g()} takes two arguments, which are \texttt{x} with a
default value of 2 and \texttt{y}, which per default is assigned to the
value of \texttt{x}. The function returns the product of \texttt{x} and
\texttt{y}. In call 2, function \texttt{g()} is called without any
arguments. In this case, x is set to 2 and y is equal to \texttt{x}. At
first glance, it is unclear if R's interpreter will take the globally
assigned value for x (here: 5) or if should take the local variable
\texttt{x}, which is 2 per default. It turns out that the default
arguments are evaluated in the local environment (that is, \emph{inside}
the actual function). Therefore, when calling \texttt{g()} with default
arguments, y is set to equal the \emph{local x} value, which is set to
the default value of 2.

\begin{Shaded}
\begin{Highlighting}[]
\KeywordTok{g}\NormalTok{(}\DataTypeTok{y=}\NormalTok{x)  }\CommentTok{# call 3}
\end{Highlighting}
\end{Shaded}

\begin{verbatim}
## [1] 10
\end{verbatim}

In \emph{call 3}, the argument for \texttt{x} is not omitted, so
\texttt{x} will get the default value 2. Further, \texttt{y} is assigned
to \texttt{x}. Differently to \emph{call 2}, the value of \texttt{y} is
evaluated when calling the function. Thus \texttt{y} is explicitly
assigned to the \emph{global x} (in this case 5).

\subsection*{Scoping II}\label{scoping-ii}
\addcontentsline{toc}{subsection}{Scoping II}

Why and how does the following code work?

\begin{Shaded}
\begin{Highlighting}[]
\NormalTok{t <-}\StringTok{ }\KeywordTok{matrix}\NormalTok{(}\DecValTok{1}\OperatorTok{:}\DecValTok{6}\NormalTok{, }\DataTypeTok{ncol =} \DecValTok{3}\NormalTok{, }\DataTypeTok{byrow =} \OtherTok{TRUE}\NormalTok{) }
\KeywordTok{t}\NormalTok{(t)}
\end{Highlighting}
\end{Shaded}

\begin{verbatim}
##      [,1] [,2]
## [1,]    1    4
## [2,]    2    5
## [3,]    3    6
\end{verbatim}

Here the variable \texttt{t} is defined to be a 2 by 3 matrix in the
\emph{Global Environment}. However, \texttt{t()} is also a function from
base R for computing the transpose of a matrix or data.frame. When
calling \texttt{t(t)}, the parentheses indicate that R should first look
for a function\texttt{t()} and skip non-function objects, and then apply
this function to a object called \texttt{t}. R searches for a function
\texttt{t()} and finds it in the \texttt{package:base} environment and
calculates the transpose of the previously defined matrix \texttt{t}.
The call \texttt{t(t)} only works because these objects with the same
name are not created within the same environment, even when taking into
consideration that one is a function and the other is a matrix object.

\subsection*{Scoping III}\label{scoping-iii}
\addcontentsline{toc}{subsection}{Scoping III}

Why do the results of t(T) and t(t) differ?

\begin{Shaded}
\begin{Highlighting}[]
\NormalTok{t <-}\StringTok{ }\ControlFlowTok{function}\NormalTok{(...) }\KeywordTok{matrix}\NormalTok{(...)}
\NormalTok{T <-}\StringTok{ }\KeywordTok{t}\NormalTok{(}\DecValTok{1}\OperatorTok{:}\DecValTok{6}\NormalTok{, }\DataTypeTok{ncol =} \DecValTok{3}\NormalTok{, }\DataTypeTok{byrow =} \OtherTok{TRUE}\NormalTok{)}
\KeywordTok{t}\NormalTok{(T)}
\end{Highlighting}
\end{Shaded}

\begin{verbatim}
##      [,1]
## [1,]    1
## [2,]    4
## [3,]    2
## [4,]    5
## [5,]    3
## [6,]    6
\end{verbatim}

In this scenarion, \texttt{t()} is a function defined in the global
environment that takes \emph{any} arguments and passes them onto the
base R \texttt{matrix()} function, which creates a matrix from a given
set of values. Further, \texttt{T} is a 2 by 3 matrix. However, applying
\texttt{t()} to \texttt{T} is the same as calling \texttt{matrix(T)}.
This returns, somewhat surprisingly, a 6 by 1 matrix. This
dimensionality distortion is due to the fact that a matrix object in R
is, under the hood, a \emph{long} one-dimensional atomic vector with a
\texttt{dim} attribute indicating the number of rows and columns. The
default value for the number of rows and columns in \texttt{matrix()}
function is 1. Therefore, when calling \texttt{matrix()} on a matrix
object (and not defining a different number of rowns or columns), the
matrix \texttt{T} gets ``unwrapped'' into the underlying one-dimensional
row vector (or a six-dimensional column vector).

The ordering {[}1,4,2,5,3,6{]} rather than {[}1,2,..,6{]} is due to
\texttt{byrow\ =\ TRUE} argument when constructing the matrix. The same
matrix could, for instance, be created by changing the dimensions of an
atomic vector, as following,

\begin{Shaded}
\begin{Highlighting}[]
\NormalTok{aMatrix <-}\StringTok{ }\KeywordTok{c}\NormalTok{(}\DecValTok{1}\NormalTok{,}\DecValTok{4}\NormalTok{,}\DecValTok{2}\NormalTok{,}\DecValTok{5}\NormalTok{,}\DecValTok{3}\NormalTok{,}\DecValTok{6}\NormalTok{)}
\KeywordTok{dim}\NormalTok{(aMatrix) <-}\StringTok{ }\KeywordTok{c}\NormalTok{(}\DecValTok{2}\NormalTok{,}\DecValTok{3}\NormalTok{)}
\end{Highlighting}
\end{Shaded}

which exemplifies the underlying nature of a matrix object in R.

\begin{Shaded}
\begin{Highlighting}[]
\NormalTok{t <-}\StringTok{ }\ControlFlowTok{function}\NormalTok{(...) }\KeywordTok{matrix}\NormalTok{(...)}
\NormalTok{t <-}\StringTok{ }\KeywordTok{t}\NormalTok{(}\DecValTok{1}\OperatorTok{:}\DecValTok{6}\NormalTok{, }\DataTypeTok{ncol =} \DecValTok{3}\NormalTok{, }\DataTypeTok{byrow =} \OtherTok{TRUE}\NormalTok{) }
\KeywordTok{t}\NormalTok{(t)}
\end{Highlighting}
\end{Shaded}

\begin{verbatim}
##      [,1] [,2]
## [1,]    1    4
## [2,]    2    5
## [3,]    3    6
\end{verbatim}

In the above scenario, the first line basically maps the base function
\texttt{matrix()} to \texttt{t()}. In the second line, \texttt{t()},
while still mapped as a function, creates a 2 by 3 matrix, which is also
mapped to \texttt{t}, overwriting the previous definition, since we
cannot have multiple objects with the same name within the same
environment. In the third line. When calling \texttt{t(t)}, R searches
for the \textbf{function} \texttt{t()} and finds the transpose function
from base R and not the already overwritten function in the Global
Environment.

\subsection*{Dynamic Lookup}\label{dynamic-lookup}
\addcontentsline{toc}{subsection}{Dynamic Lookup}

Explain the results of the five function calls and why the rm function
in line 1 is important.

\begin{Shaded}
\begin{Highlighting}[]
\KeywordTok{rm}\NormalTok{(}\DataTypeTok{list =} \KeywordTok{ls}\NormalTok{(}\DataTypeTok{all.names =} \OtherTok{TRUE}\NormalTok{))}
\NormalTok{f <-}\StringTok{ }\ControlFlowTok{function}\NormalTok{(x, }\DataTypeTok{y =}\NormalTok{ x }\OperatorTok{+}\StringTok{ }\DecValTok{1}\NormalTok{) x }\OperatorTok{+}\StringTok{ }\NormalTok{y }
\NormalTok{x <-}\StringTok{ }\DecValTok{3}
\KeywordTok{f}\NormalTok{(}\DecValTok{2}\NormalTok{) }\CommentTok{# call 1}
\NormalTok{## [1] 5}
\NormalTok{x <-}\StringTok{ }\DecValTok{5}
\KeywordTok{f}\NormalTok{(}\DecValTok{2}\NormalTok{) }\CommentTok{# call 2}
\NormalTok{## [1] 5}

\NormalTok{f <-}\StringTok{ }\ControlFlowTok{function}\NormalTok{(}\DataTypeTok{y =}\NormalTok{ x }\OperatorTok{+}\StringTok{ }\DecValTok{1}\NormalTok{) x }\OperatorTok{+}\StringTok{ }\NormalTok{y }
\NormalTok{x <-}\StringTok{ }\DecValTok{3}
\KeywordTok{f}\NormalTok{(}\DecValTok{2}\NormalTok{) }\CommentTok{# call 3}
\NormalTok{## [1] 5}
\NormalTok{x <-}\StringTok{ }\DecValTok{5}
\KeywordTok{f}\NormalTok{(}\DecValTok{2}\NormalTok{) }\CommentTok{# call 4}
\NormalTok{## [1] 7}
\KeywordTok{f}\NormalTok{()  }\CommentTok{# call 5}
\NormalTok{## [1] 11}
\end{Highlighting}
\end{Shaded}

\begin{description}
\tightlist
\item[Call 1 and 2]
This function requires one argument, \texttt{x}, but also accepts a
second argument, \texttt{y}. It returns the sum of \texttt{x} and
\texttt{y}, with \texttt{y} equals \texttt{x\ +\ 1} per default.

This exemplifies the lazy evaluation property in R. By the time the
function is created \texttt{x} doesn't exist and only when the function
\texttt{f(2)} is called, 2 is passed as an argument for \texttt{x} and
and \texttt{y} gets assigned to the expression \texttt{x\ +\ 1} from the
newly created function environment.) The global value of \texttt{x} does
not affect the function. Thus, call 1 and 2 returns the same result. One
can affirm that \texttt{f()} is (weakly) self contained. That means, the
values from global or parent environments don't affect the output of the
function
\item[Call 3 and 4]
This function accepts only one argument, \texttt{y}, but a second
variable, \texttt{x}, is required for it to work. Since this variable is
not created within the function environment, R will \emph{go up} the
search path looking for a variable called \texttt{x}. Even when passing
the same argument for \texttt{y}, different values of \texttt{x} will
yield different results, which can be seen on the results of call 3 and
call 4. This function is, therefore, not self-contained.
\item[Call 5]
Since the default value for \texttt{y} is \texttt{x\ +\ 1} and no
argument is passed in the last call, the function returns the value for
\texttt{x\ +\ (x\ +\ 1)}, where \texttt{x} is the defined in the global
environment.
\item[rm function]
The rm function is useful to make sure that previously assigned values
do not generate unexpected results. Nevertheless, it is probably better
to make sure this line is not necessary, reducing the risks of getting
one's computer burned\footnote{see discussion on
  \url{https://twitter.com/hadleywickham/status/940021008764846080}}.
\end{description}


\end{document}
