\documentclass[12,]{article}
\usepackage{lmodern}
\usepackage{amssymb,amsmath}
\usepackage{ifxetex,ifluatex}
\usepackage{fixltx2e} % provides \textsubscript
\ifnum 0\ifxetex 1\fi\ifluatex 1\fi=0 % if pdftex
  \usepackage[T1]{fontenc}
  \usepackage[utf8]{inputenc}
\else % if luatex or xelatex
  \ifxetex
    \usepackage{mathspec}
  \else
    \usepackage{fontspec}
  \fi
  \defaultfontfeatures{Ligatures=TeX,Scale=MatchLowercase}
\fi
% use upquote if available, for straight quotes in verbatim environments
\IfFileExists{upquote.sty}{\usepackage{upquote}}{}
% use microtype if available
\IfFileExists{microtype.sty}{%
\usepackage{microtype}
\UseMicrotypeSet[protrusion]{basicmath} % disable protrusion for tt fonts
}{}
\usepackage[margin=1in]{geometry}
\usepackage{hyperref}
\hypersetup{unicode=true,
            pdftitle={Programmieren mit R: Seminararbeit 3},
            pdfauthor={Daniyar Akhmetov (5127348); Marcelo Rainho Avila (4679876); Xuan Son Le (4669361)},
            pdfborder={0 0 0},
            breaklinks=true}
\urlstyle{same}  % don't use monospace font for urls
\usepackage{color}
\usepackage{fancyvrb}
\newcommand{\VerbBar}{|}
\newcommand{\VERB}{\Verb[commandchars=\\\{\}]}
\DefineVerbatimEnvironment{Highlighting}{Verbatim}{commandchars=\\\{\}}
% Add ',fontsize=\small' for more characters per line
\usepackage{framed}
\definecolor{shadecolor}{RGB}{248,248,248}
\newenvironment{Shaded}{\begin{snugshade}}{\end{snugshade}}
\newcommand{\KeywordTok}[1]{\textcolor[rgb]{0.13,0.29,0.53}{\textbf{#1}}}
\newcommand{\DataTypeTok}[1]{\textcolor[rgb]{0.13,0.29,0.53}{#1}}
\newcommand{\DecValTok}[1]{\textcolor[rgb]{0.00,0.00,0.81}{#1}}
\newcommand{\BaseNTok}[1]{\textcolor[rgb]{0.00,0.00,0.81}{#1}}
\newcommand{\FloatTok}[1]{\textcolor[rgb]{0.00,0.00,0.81}{#1}}
\newcommand{\ConstantTok}[1]{\textcolor[rgb]{0.00,0.00,0.00}{#1}}
\newcommand{\CharTok}[1]{\textcolor[rgb]{0.31,0.60,0.02}{#1}}
\newcommand{\SpecialCharTok}[1]{\textcolor[rgb]{0.00,0.00,0.00}{#1}}
\newcommand{\StringTok}[1]{\textcolor[rgb]{0.31,0.60,0.02}{#1}}
\newcommand{\VerbatimStringTok}[1]{\textcolor[rgb]{0.31,0.60,0.02}{#1}}
\newcommand{\SpecialStringTok}[1]{\textcolor[rgb]{0.31,0.60,0.02}{#1}}
\newcommand{\ImportTok}[1]{#1}
\newcommand{\CommentTok}[1]{\textcolor[rgb]{0.56,0.35,0.01}{\textit{#1}}}
\newcommand{\DocumentationTok}[1]{\textcolor[rgb]{0.56,0.35,0.01}{\textbf{\textit{#1}}}}
\newcommand{\AnnotationTok}[1]{\textcolor[rgb]{0.56,0.35,0.01}{\textbf{\textit{#1}}}}
\newcommand{\CommentVarTok}[1]{\textcolor[rgb]{0.56,0.35,0.01}{\textbf{\textit{#1}}}}
\newcommand{\OtherTok}[1]{\textcolor[rgb]{0.56,0.35,0.01}{#1}}
\newcommand{\FunctionTok}[1]{\textcolor[rgb]{0.00,0.00,0.00}{#1}}
\newcommand{\VariableTok}[1]{\textcolor[rgb]{0.00,0.00,0.00}{#1}}
\newcommand{\ControlFlowTok}[1]{\textcolor[rgb]{0.13,0.29,0.53}{\textbf{#1}}}
\newcommand{\OperatorTok}[1]{\textcolor[rgb]{0.81,0.36,0.00}{\textbf{#1}}}
\newcommand{\BuiltInTok}[1]{#1}
\newcommand{\ExtensionTok}[1]{#1}
\newcommand{\PreprocessorTok}[1]{\textcolor[rgb]{0.56,0.35,0.01}{\textit{#1}}}
\newcommand{\AttributeTok}[1]{\textcolor[rgb]{0.77,0.63,0.00}{#1}}
\newcommand{\RegionMarkerTok}[1]{#1}
\newcommand{\InformationTok}[1]{\textcolor[rgb]{0.56,0.35,0.01}{\textbf{\textit{#1}}}}
\newcommand{\WarningTok}[1]{\textcolor[rgb]{0.56,0.35,0.01}{\textbf{\textit{#1}}}}
\newcommand{\AlertTok}[1]{\textcolor[rgb]{0.94,0.16,0.16}{#1}}
\newcommand{\ErrorTok}[1]{\textcolor[rgb]{0.64,0.00,0.00}{\textbf{#1}}}
\newcommand{\NormalTok}[1]{#1}
\usepackage{graphicx,grffile}
\makeatletter
\def\maxwidth{\ifdim\Gin@nat@width>\linewidth\linewidth\else\Gin@nat@width\fi}
\def\maxheight{\ifdim\Gin@nat@height>\textheight\textheight\else\Gin@nat@height\fi}
\makeatother
% Scale images if necessary, so that they will not overflow the page
% margins by default, and it is still possible to overwrite the defaults
% using explicit options in \includegraphics[width, height, ...]{}
\setkeys{Gin}{width=\maxwidth,height=\maxheight,keepaspectratio}
\IfFileExists{parskip.sty}{%
\usepackage{parskip}
}{% else
\setlength{\parindent}{0pt}
\setlength{\parskip}{6pt plus 2pt minus 1pt}
}
\setlength{\emergencystretch}{3em}  % prevent overfull lines
\providecommand{\tightlist}{%
  \setlength{\itemsep}{0pt}\setlength{\parskip}{0pt}}
\setcounter{secnumdepth}{5}
% Redefines (sub)paragraphs to behave more like sections
\ifx\paragraph\undefined\else
\let\oldparagraph\paragraph
\renewcommand{\paragraph}[1]{\oldparagraph{#1}\mbox{}}
\fi
\ifx\subparagraph\undefined\else
\let\oldsubparagraph\subparagraph
\renewcommand{\subparagraph}[1]{\oldsubparagraph{#1}\mbox{}}
\fi

%%% Use protect on footnotes to avoid problems with footnotes in titles
\let\rmarkdownfootnote\footnote%
\def\footnote{\protect\rmarkdownfootnote}

%%% Change title format to be more compact
\usepackage{titling}

% Create subtitle command for use in maketitle
\newcommand{\subtitle}[1]{
  \posttitle{
    \begin{center}\large#1\end{center}
    }
}

\setlength{\droptitle}{-2em}
  \title{Programmieren mit R: Seminararbeit 3}
  \pretitle{\vspace{\droptitle}\centering\huge}
  \posttitle{\par}
  \author{Daniyar Akhmetov (5127348) \\ Marcelo Rainho Avila (4679876) \\ Xuan Son Le (4669361)}
  \preauthor{\centering\large\emph}
  \postauthor{\par}
  \predate{\centering\large\emph}
  \postdate{\par}
  \date{Abgabedatum: 08.02.2018}


\begin{document}
\maketitle

{
\setcounter{tocdepth}{3}
\tableofcontents
}
\newpage

\section{\texorpdfstring{Part I: \emph{Linear regression} (15
points)}{Part I: Linear regression (15 points)}}\label{part-i-linear-regression-15-points}

\subsection{\texorpdfstring{\emph{Raw
implementation}}{Raw implementation}}\label{raw-implementation}

\begin{Shaded}
\begin{Highlighting}[]
\NormalTok{linModEst <-}\StringTok{ }\ControlFlowTok{function}\NormalTok{(x,y) \{}
\NormalTok{  ## compute temp=(x'x)^(-1) with x'=t(x)}
\NormalTok{  temp <-}\StringTok{ }\KeywordTok{crossprod}\NormalTok{(x,}\DataTypeTok{y =} \OtherTok{NULL}\NormalTok{) }\CommentTok{#compute (x'x)}
\NormalTok{  temp <-}\StringTok{ }\KeywordTok{solve}\NormalTok{(temp) }\CommentTok{#compute (x'x)^(-1)}
  
\NormalTok{  ## compute beta}
\NormalTok{  temp1 <-}\StringTok{ }\KeywordTok{crossprod}\NormalTok{(x,y) }\CommentTok{#compute(x'y)}
\NormalTok{  beta <-}\StringTok{ }\KeywordTok{crossprod}\NormalTok{(}\KeywordTok{t}\NormalTok{(temp),temp1) }\CommentTok{#compute beta}
  
\NormalTok{  ## calculate degree of freedom}
\NormalTok{  df <-}\StringTok{ }\KeywordTok{nrow}\NormalTok{(x) }\OperatorTok{-}\StringTok{ }\KeywordTok{ncol}\NormalTok{(x)}
  
\NormalTok{  ## calculate sigma^2}
\NormalTok{  SSR <-}\StringTok{ }\DecValTok{0}
\NormalTok{  help <-}\StringTok{ }\DecValTok{0}
  \ControlFlowTok{for}\NormalTok{ (i }\ControlFlowTok{in}\NormalTok{ (}\DecValTok{1}\OperatorTok{:}\KeywordTok{nrow}\NormalTok{(x))) \{}
\NormalTok{    help <-}\StringTok{ }\NormalTok{(x[i,]) }\OperatorTok\StringTok{ }\NormalTok{beta}
\NormalTok{    SSR <-}\StringTok{ }\NormalTok{SSR }\OperatorTok{+}\StringTok{ }\NormalTok{(y[i] }\OperatorTok{-}\StringTok{ }\NormalTok{help)}\OperatorTok{^}\DecValTok{2}
\NormalTok{  \}}
\NormalTok{  sigma_power_}\DecValTok{2}\NormalTok{ <-}\StringTok{ }\KeywordTok{as.double}\NormalTok{(SSR }\OperatorTok{/}\StringTok{ }\NormalTok{df)}
\NormalTok{  sigma <-}\StringTok{ }\KeywordTok{sqrt}\NormalTok{(sigma_power_}\DecValTok{2}\NormalTok{)}
  
  \CommentTok{# calculate covariance matrix}
\NormalTok{  vcov <-}\StringTok{ }\NormalTok{sigma_power_}\DecValTok{2} \OperatorTok{*}\StringTok{ }\NormalTok{temp}
  
  \CommentTok{# return results}
  \KeywordTok{list}\NormalTok{(}\DataTypeTok{beta =}\NormalTok{ beta, }\DataTypeTok{vcov =}\NormalTok{ vcov, }\DataTypeTok{sigma =}\NormalTok{ sigma, }\DataTypeTok{df =}\NormalTok{ df)}
\NormalTok{\}}

\KeywordTok{data}\NormalTok{(cats, }\DataTypeTok{package =} \StringTok{"MASS"}\NormalTok{)}
\KeywordTok{linModEst}\NormalTok{(}\DataTypeTok{x =} \KeywordTok{cbind}\NormalTok{(}\DecValTok{1}\NormalTok{, cats}\OperatorTok{$}\NormalTok{Bwt,}\KeywordTok{as.numeric}\NormalTok{(cats}\OperatorTok{$}\NormalTok{Sex) }\OperatorTok{-}\StringTok{ }\DecValTok{1}\NormalTok{), }
          \DataTypeTok{y =}\NormalTok{ cats}\OperatorTok{$}\NormalTok{Hwt)}
\end{Highlighting}
\end{Shaded}

\begin{verbatim}
## $beta
##             [,1]
## [1,] -0.41495263
## [2,]  4.07576892
## [3,] -0.08209684
## 
## $vcov
##             [,1]        [,2]        [,3]
## [1,]  0.52900070 -0.20504763  0.06563743
## [2,] -0.20504763  0.08690026 -0.04696312
## [3,]  0.06563743 -0.04696312  0.09244480
## 
## $sigma
## [1] 1.457138
## 
## $df
## [1] 141
\end{verbatim}

\begin{itemize}
\tightlist
\item
  By adding a new column with all values of 1, we transformed the given
  matrix x into X.\\
\item
  \texttt{crossprod(x,\ y\ =\ NULL)} is equal to
  \texttt{t(x)\ \%*\%\ y}. Because \texttt{y\ =\ NULL} is taken to be
  the same matrix as \texttt{x}, the result will be
  \texttt{t(x)\ \%*\%\ x}. To find the inverse matrix of
  \texttt{x\textquotesingle{}x} we used the function \texttt{solve()}.\\
\item
  x is given as cbind bla, because the first column is used for
  \texttt{ßo}
\end{itemize}

\subsection{\texorpdfstring{\emph{Check
equivalent}}{Check equivalent}}\label{check-equivalent}

\begin{Shaded}
\begin{Highlighting}[]
\NormalTok{Hwt <-}\StringTok{ }\NormalTok{cats}\OperatorTok{$}\NormalTok{Hwt}
\NormalTok{Bwt <-}\StringTok{ }\NormalTok{cats}\OperatorTok{$}\NormalTok{Bwt}
\NormalTok{Sex <-}\StringTok{ }\KeywordTok{as.numeric}\NormalTok{(cats}\OperatorTok{$}\NormalTok{Sex) }\OperatorTok{-}\StringTok{ }\DecValTok{1}

\NormalTok{lm_cat <-}\StringTok{ }\KeywordTok{lm}\NormalTok{(Hwt }\OperatorTok{~}\StringTok{ }\NormalTok{Bwt }\OperatorTok{+}\StringTok{ }\NormalTok{Sex, }\DataTypeTok{data =}\NormalTok{ cats)}
\KeywordTok{coef}\NormalTok{(lm_cat)}
\end{Highlighting}
\end{Shaded}

\begin{verbatim}
## (Intercept)         Bwt        SexM 
## -0.41495263  4.07576892 -0.08209684
\end{verbatim}

\begin{Shaded}
\begin{Highlighting}[]
\KeywordTok{vcov}\NormalTok{(lm_cat)}
\end{Highlighting}
\end{Shaded}

\begin{verbatim}
##             (Intercept)         Bwt        SexM
## (Intercept)  0.52900070 -0.20504763  0.06563743
## Bwt         -0.20504763  0.08690026 -0.04696312
## SexM         0.06563743 -0.04696312  0.09244480
\end{verbatim}

The same!!

\subsection{\texorpdfstring{\emph{Extend
implementation}}{Extend implementation}}\label{extend-implementation}

\begin{Shaded}
\begin{Highlighting}[]
\NormalTok{linMod <-}\StringTok{ }\ControlFlowTok{function}\NormalTok{(formula, data)\{}
  \CommentTok{# still no idea :((}
  \KeywordTok{lm}\NormalTok{(formula, data) }\CommentTok{# wäre schön ^^}
\NormalTok{\}}

\KeywordTok{linMod}\NormalTok{(Hwt }\OperatorTok{~}\StringTok{ }\NormalTok{Bwt }\OperatorTok{+}\StringTok{ }\NormalTok{Sex, }\DataTypeTok{data =}\NormalTok{ cats)}
\end{Highlighting}
\end{Shaded}

\begin{verbatim}
## 
## Call:
## lm(formula = formula, data = data)
## 
## Coefficients:
## (Intercept)          Bwt         SexM  
##     -0.4150       4.0758      -0.0821
\end{verbatim}

\section{PartII: S3 for linear
models}\label{partii-s3-for-linear-models}


\end{document}
